
\chapter{Introduction}

\selectlanguage{english}%
With the advancement of information technology in modern generation,
the evolution of digital era has become more mature in the sense of
effectiveness and easiness for societies. They can sell and buy goods,
conduct banking activities and even participate in political activities
such as election through online. Trusted entities such as financial
institutions generally offer their products and services to the public
through the Internet. Furthermore, modern technology has greatly impacted
our society in different ways, such as the way we communicate with
each other. Nowadays, we are no longer need to use a computer to send
an email, we can just use our smart phone, which we carry every day
in our pockets, with internet connectivity to send an email. As a
result, human society has been utilizing technology means such as
emails, websites, online payment system, social networks to achieve
their tasks efficiently, affordable and more relevant. However, the
advancement in information and communication technology has been a
double-edged sword. As the internet increasingly become more accessible,
people tend to share more about themselves and as a consequence, it
becomes easier to get personal information about someone on the Internet.
Cyber criminals see this opportunity as a way to manipulate consumers
and exploit their confidential information such as usernames, passwords,
bank account information, credit card or social security numbers.
Personalized information about someone such as email addresses, phone
numbers, birth dates, relationships or work place might be obtained
from the internet. Consequently, cyber criminals can compose an attack
in a personalized way to persuade intended victims to grant their
malicious requests.

One particular type of cyber crimes is called phishing. Many possible
incentives that drive phishing attacks such as illicit corporate espionage,
political power, and the most common incentive of phishing attacks
is financial benefits. The attacker generally masquerades legitimate
institutions to trick users into disclosing personal, financial or
computer account information \citep{jakobsson:2006}. The attacker
can then use this information for criminal activities such as identity
theft or fraud. To manipulate unsuspecting victims, the attacker often
uses emails and websites as the techniques to execute the attacks
\citep{jakobsson:2006}\citep{dhamija2006phishing}. The practice
of utilizing emails and websites is indeed useful as communication
media, however, they can also accommodate deceptive attacks such as
phishing as a form of social engineering and deception \citep{jakobsson:2006}\citep{blythe2011f}\citep{dhamija2006phishing}\citep{james:2005}\citep{jagatic2007social}.
Social engineering involves the techniques used to deceive people
in order to get compliance and response by specific actions which
will disclose their sensitive information, such as replying to a fake
email or clicking a link within an email \citep{mitnik:2001}. Moreover,
phishers often use persuasion techniques to manipulate potential victims
to engage in certain emotions such as excitement and fear as well
as interpersonal relationship, such as trust and commitments to divert
users' attention \citep{workman:2008}. Such persuasive influence
might be delivered through phone calls, text messages, private messages
or emails as ways to distract recipients' decisions. 

\selectlanguage{american}%

\section{Problem statement}

\selectlanguage{english}%
Countermeasures against phishing attacks via email can be technical
or non technical. One of technical approaches to detect phishing email
is achieved by discriminating phishing and original email based on
its structural properties using machine learning \citep{chandrasekaran:2006}.
One of non technical approaches to defend against phishing attacks
is to make people aware of the threats. Security awareness concerning
phishing attacks might be achieved by embedded training methods that
teach people about phishing during their normal use of email \citep{kumaraguru2007protecting}. 

Common characteristics of a phishing email include structural properties
such as misleading hyperlinks and misleading header information \citep{zhang:2006,zhang:2007}.
However, to make a phishing email efficient, its content requires
the intended victim to urgently act upon it, for example, an email
that informs about account termination if the recipient does not respond
or perform an action within a limited time. In order to obtain compliance
from the recipient in a phishing email, persuasion is used as underlying
techniques to get a positive response from an intended victim \citep{workman:2008}.

The success of a phishing attack through distributed emails is determined
by the response of the unsuspecting recipients. User decisions to
click a link or open an attachment in an email might be influenced
by how strong a phisher can persuade the intended victim. Phishers
often misuse persuasion techniques to get positive responses from
the recipients. Unfortunately, not many studies have included persuasion
techniques as the important aspects in phishing emails. Consequently,
not many people are aware of the existence of persuasion in the emails
they are getting. Current phishing email countermeasures are greatly
relying on the technical aspect alone rather than integrating the
technical aspect with psychological aspect such as persuasion. Based
on Cialdini's persuasion principle, there are six basic tendencies
of human behavior that can generate a positive response; reciprocation,
consistency, social proof, likeability, authority and scarcity \citep{cialdini:2001}.
As we mention earlier, persuasion techniques can also be exploited
by the phishers to get positive responses from potential victims.
Based on this reasoning, we will conduct a quick scan of the existing
studies regarding persuasion techniques in phishing emails.

We performed a combination of query strings such as ``phishing''
and ``persuasion'' to search based on title, abstract, and keyword
fields in Scopus database and we also search based on topic in Web
of Science database. As we will adopt the concept of persuasion based
on Cialdini's six principle, we search ``phishing'' in TAK(Title,
Abstract, Keywords) fields and ``Cialdini'' in the References field.
From \autoref{tab:Query-searches-in}, we can see that there are two
papers that occurred in both databases \citep{wright2014research,kaivanto2014effect}
and one paper occurred in both queries \citep{wright2014research}.
However, Sharma's paper is retracted for having non original materials,
hence, we will not conduct a review on it. We will describe what the
remaining papers do and how they are different from our research in
the following points:

\begin{table}
\begin{centering}
\begin{tabular}{>{\centering}p{3cm}>{\centering}p{3cm}>{\centering}p{3cm}}
\toprule 
\selectlanguage{american}%
\textbf{\scriptsize{}Keywords}\selectlanguage{american}%
 & \selectlanguage{american}%
\textbf{\scriptsize{}Scopus}\selectlanguage{american}%
 & \selectlanguage{american}%
\textbf{\scriptsize{}Web of science}\selectlanguage{american}%
\tabularnewline
\midrule
\midrule 
{\scriptsize{}TITLE-ABS-KEY ( phishing persuasion ) } & \selectlanguage{american}%
{\scriptsize{}\citep{kaivanto2014effect,wright2014research,kim2013understanding,blythe2011f,sharma2010anatomy,Ghorbani20101}}\selectlanguage{american}%
 & \selectlanguage{american}%
{\scriptsize{}\citep{wright2014research,kim2013understanding,workman:2008}}\selectlanguage{american}%
\tabularnewline
\midrule 
\selectlanguage{american}%
{\scriptsize{}( TITLE-ABS-KEY ( phishing ) AND REF ( cialdini ) })\selectlanguage{american}%
 & \selectlanguage{american}%
{\scriptsize{}\citep{wright2014research,krombholz2013social,herzberg2013forcing,vishwanath2011people,kawakami2010development}}\selectlanguage{american}%
 & \selectlanguage{american}%
\selectlanguage{american}%
\tabularnewline
\bottomrule
\end{tabular}\protect\caption{\selectlanguage{american}%
\label{tab:Query-searches-in}Query searches in Scopus and Web of
Science\selectlanguage{american}%
}

\par\end{centering}

\selectlanguage{american}%
\selectlanguage{american}%
\end{table}


\selectlanguage{american}%
\ 
\selectlanguage{english}%
\begin{itemize}
\item As observed by Kavianto, overall security risk in an organization
depends on its individual\textquoteright s behavioral decision to
respond to security threats such as phishing attacks. Kavianto also
found that individual's behavioral decision can be segregated into
two level; tradeoffs between uncertainties, losses and benefits, and
their susceptibility to persuasion techniques \citep{cialdini:2001}
and emotion \citep{kaivanto2014effect}. Kavianto develops a model
of the individual's behavioral decisions in aggregate levels \citep{kaivanto2014effect}.
The outcome of the study is the possibility of incorporating individual-level
behavioral biases into the analysis of system level risk (i.e. network
security risk). Although Kavianto identifies that successful deception
can be linked with how persuasion techniques used by the perpetrator,
there is no actual data of phishing emails have been analyzed.
\item Blythe, et al. were conducting four methods to show how people are
often susceptible to phishing attacks \citep{blythe2011f}. These
four methods are; content analysis, online surveys, interview with
blind email users and literary analysis \citep{blythe2011f}. Content
analysis suggests that phishing email is advancing with better spelling,
grammar and supported by the present of visual graphic such as logos.
Online survey shows that while their participants are computer literate,
the participants did not always successful in detecting phishing,
even more so with the present of logo. Blythe, et al. found that blind
email users are more attentive to the context of the email that is
presented \citep{blythe2011f}. Thus, the detection of phishing by
blind email users is higher than non-disabled email users. As it became
clear that careful reading is the core process of identifying phishing
emails, Blythe, et al. then consider a phishing email as a literature
\citep{blythe2011f}. Literary analysis shows that the phishers who
imitate personalized email from banking and security services, makes
phishing emails remain successful, as they take part into people\textquoteright s
anxieties in terms of the content of the email itself. Although Blythe,
et al. conducted content analysis and found that literary analysis
claimed to be ``very'' persuasive, but they did not conduct the
content analysis based on persuasion techniques. Instead, Blythe,
et al. conducted the content analysis based on the structural properties
such as sender, grammatical error, logos and style.
\item Ghorbani, et al. argued that some general approaches designed by attackers
to obtain sensitive information to exploit a computer system are using
social engineering \citep{Ghorbani20101}. Social engineering includes
aggressive persuasion and interpersonal skills to obtain unauthorized
access to a system. Moreover, Ghorbani, et al. discussed network attack
taxonomy, probes, privilege escalation attacks, Denial of Service
(DOS) attacks and Distributed Denial of Services (DDoS) attacks, worm
and routing attacks \citep{Ghorbani20101}. However, their study only
discussed these network attacks in considerable detail without addressing
persuasion theory and little explanation in terms of phishing emails.
\item Krombholz, et al. paper provides a taxonomy of well known social engineering
attacks and studies the overview of advanced social engineering attacks
on knowledge worker \citep{krombholz2013social}. What they meant
by knowledge worker here is the worker that characterized knowledge
as one's capital. The paper used Cialdini's persuasion principles
as the background study of social engineering. The taxonomy was classified
based on attack channel (i.e. emails, instant messenger, social network,
etc.), the operator (i.e. human, software), different types of social
engineering (i.e. phishing, shoulder surfing, dumpster diving, etc.)
and specific attack scenarios. Krombholz, et al. showcase the overview
of social engineering attacks by creating a taxonomy to support further
development of social engineering attack countermeasures \citep{krombholz2013social}. 
\item Herzberg, et al. tested the effectiveness of different defense mechanisms
that use forcing and negative training functions \citep{herzberg2013forcing}.
Their methods was by using an online exercise submission system called
``Submit'' to simulate phishing attacks and it involves the population
of \textasciitilde{}400 students and two years of observation. Their
outcome claimed that forcing and negative training functions are very
effective in both privation and detection. However, their defense
mechanisms do not consider on the persuasion techniques at all nor
analyzing data from the real phishing emails.
\item Vishwanath, et al. tested the individual differences in phishing vulnerability
within integrated information processing model \citep{vishwanath2011people}.
The model focuses on four contextual factors: the individual level
of involvement, domain specific knowledge, technological efficacy,
and email load. The method involves the total number of 161 sample.
The outcome of their study is that to show the model can be used as
an insight into how individuals get phished. However, they do not
consider persuasion principles on their contextual factors nor analyze
the real phishing attacks. This suggests that persuasion techniques
are not part of the important role to determined the success of phishing
attacks.
\item Kawakami, et al. developed an e-learning system which functions to
use animation for information security education \citep{kawakami2010development}.
The paper only mentions the commitment and consistency based on Cialdini's
principles to be used as influence for people to take their e-learning
system as a security education.
\item An interesting paper from Wright, et al. which analyze how people
influenced by the Cialdini's six persuasion principles in phishing
messages \citep{wright2014research}\citep{cialdini:2001}. However,
the procedures of the study involved with creating phishing messages
which represent persuasion principles and test them to 2,600 participants
and see how many would respond \citep{wright2014research}. The outcome
of the study is that; liking, social proof, scarcity and reciprocation
do increase the likelihood of recipients will respond to phishing
emails \citep{wright2014research}. Despite the fact that Wright,
et al. were using the same persuasion principles as our study \citep{cialdini:2001},
Wright, et al. tried to find the implication of persuasion principles
from the users' perspectives. Although we find that the direction
of their paper is different, it can be a complimentary to our study.
\item Workman proposed a peripheral route of persuasion model and conducted
a behavioral study based on this model \citep{workman:2008}. The
model is created to relate basic human factors to respond to persuasion
techniques based on Cialdini\textquoteright s six principle \citep{cialdini:2001}.
These factors include; normative commitment, continuance commitment,
affective commitment, trust, fear, and reactance. Workman created
six hypotheses to investigate whether the participants who are prone
to these factors, exhibits a higher risk to phishing attacks \citep{workman:2008}.
Based on Workman's measurement, six of the hypotheses are accepted.
The data was obtained by a questionnaire and objective observation
involving 850 participants. In the end, the total of 612 participant
has responded. The conclusion is that, the participants who have the
tendency of these factors, are more vulnerable to phishing attacks
\citep{workman:2008}. For instance, one of Workman's hypotheses stated
that ``People who are more reactance, will succumb to social engineering
more frequently than those who are more resistance''. This suggests
that Workman have tried to measure the implication of persuasion principles
from the users' perspectives as well. 
\item We found only one paper which was similar to our study \citep{kim2013understanding}.
Apart from different dataset, Kim, et al. conducted a content analysis
of phishing emails based on message argument quality rather than Cialdini's
six persuasion principles \citep{kim2013understanding}\citep{cialdini:2001}.
Besides that, Kim, et al. did not relate structural phishing properties
such as URLs, attachments, usage of HTML, targeted sector and reason
used with their persuasion theory. The message argument quality includes;
rational appeals, emotional appeals, motivational appeals and time
pressure \citep{kim2013understanding}. The reasoning of rational
appeals determined by direct evidence of causality between events.
For example, \textquotedblleft a few days ago our online banking security
team observed invalid logins to customer accounts. Thus, you are required
to re-confirm your online access for account verification\textquotedblright .
This can be mapped as reciprocation based on Cialdini's persuasion
principles \citep{cialdini:2001}. Emotional appeal defined by fear,
sadness, guilt, anger, happiness, affection and humor. In our study
this can be mapped into authority or likeability. Time pressure identified
by the limited amount of time the recipient has, to respond to a phishing
email. This also can be mapped into scarcity principle. Moreover,
one of the factors representing motivational appeals is that ``the
need of belongingness'' which also can be portrayed as the need of
being part of a group to forms bond with others. Based on our understanding,
we can map motivational appeals as social proof principle. However,
there is no conception of message argument quality that can be mapped
to consistency principle. \autoref{tab:A-map-of} indicates our mapping
from message argument quality into Cialdini's persuasion principles
\citep{cialdini:2001}. One interesting result from the study is that,
the number of time pressure emails (42\%, n=285), is not as high as
they expected. The study concludes that phishers indeed incorporate
rational, emotional and motivational appeals in their dataset. However,
the conception of persuasion theory adopted by Kim, et al. is different
from our study. We argue that diversity of persuasion theory needs
to be incorporated to achieve an objective conclusion. 
\end{itemize}
\begin{table}[H]
\selectlanguage{american}%


\selectlanguage{english}%
\begin{centering}
{\scriptsize{}}%
\begin{tabular}{cc}
\toprule 
\selectlanguage{american}%
\textbf{\scriptsize{}Message argument quality}\selectlanguage{american}%
 & \selectlanguage{american}%
\textbf{\scriptsize{}Cialdini's persuasion principles}\selectlanguage{american}%
\tabularnewline
\midrule
\midrule 
\selectlanguage{american}%
{\scriptsize{}Rational appeals}\selectlanguage{american}%
 & \selectlanguage{american}%
{\scriptsize{}Reciprocation}\selectlanguage{american}%
\tabularnewline
\midrule 
\selectlanguage{american}%
{\scriptsize{}Emotional appeals}\selectlanguage{american}%
 & \selectlanguage{american}%
{\scriptsize{}Authority and Likeability}\selectlanguage{american}%
\tabularnewline
\midrule 
\selectlanguage{american}%
{\scriptsize{}Motivational appeals}\selectlanguage{american}%
 & \selectlanguage{american}%
{\scriptsize{}Social proof}\selectlanguage{american}%
\tabularnewline
\midrule 
\selectlanguage{american}%
{\scriptsize{}Time pressure}\selectlanguage{american}%
 & \selectlanguage{american}%
{\scriptsize{}Scarcity}\selectlanguage{american}%
\tabularnewline
\bottomrule
\end{tabular}\protect\caption{\selectlanguage{american}%
\label{tab:A-map-of}A map of message argument quality \citep{kim2013understanding}
to Cialdini's persuasion principles \foreignlanguage{english}{\citep{cialdini:2001}}\selectlanguage{american}%
}

\par\end{centering}

\selectlanguage{american}%
\selectlanguage{american}%
\end{table}


Based on the review, we can say that the studies conducted by Workman
\citep{workman:2008} and Wright, et al. \citep{wright2014research}
are similar. Both of them have tried to find the implication of persuasion
principles from the users' perspectives, which can be a complementary
to our study. We have conducted individual investigations on all papers
found in both databases, the finding shows not only the different
of measurement instruments and methods but also the foundation of
persuasion theory between the current studies and our study. The low
number of results from the query searches also indicate not many academic
researchers that study persuasion techniques and phishing area. Therefore,
a real world analysis of phishing emails characterization based on
persuasion techniques is needed to bridge this gap. This characterization
can show to what extent the persuasion techniques are used in phishing
emails. Our research fills the void as a milestone towards countermeasures
against phishing attacks with an insight of psychological aspect.


\section{Research goal}

The main goal of this research is to characterize phishing email properties
considering persuasion principles by finding the association between
generic properties and persuasion principles. These generic properties
consist of phishing email structural properties or features based
on the literature survey findings. Each of these properties and each
of the persuasion principle will be introduced as a variable in our
methodology. We will look for frequency and relationship involving
these variables. This relationship can be used to show a different
perspective of phishing email characteristics considering the persuasive
elements within its content. The analysis of persuasion principles
in phishing emails also can be used to generate a new method in an
automated way of detecting phishing emails as one of the primary delivery
techniques of phishing attacks.


\section{Research Questions}

To be able to meet the goal, we formulated two main research questions
as follows:
\begin{itemize}
\item RQ1: What are the characteristics of phishing emails?
\item RQ2: To what extent are persuasion principles used in phishing emails?
\end{itemize}
\selectlanguage{american}%
Several aspects of phishing email characteristics and hypotheses related
to the research questions will be addressed in detail in \autoref{chap:research-questions}.

\selectlanguage{english}%

\section{Structures}

This research project is structured as follows: 

Chapter 2 describes background and literature reviews about phishing
in general. The subsections include; a general understanding of what
is phishing in terms of history and definition, an overview of its
damage in terms of money, an exploration of its modus operandi based
on phishing stages or phases, general phishing countermeasures and
lastly the human factor in phishing. 

In chapter 3, we present the rationale of our main research questions
and hypotheses. It includes what aspect to be considered to answer
the characteristics of phishing emails based on persuasion principles
in the dataset and the motivation of our hypotheses to support our
research questions.

In chapter 4, we will discuss our main data analysis and results.
It includes the details of research methodology that we conducted
as well as the results of our analysis.

Lastly, in chapter 5 we will present our discussion and conclusion
of the research project, how the research questions are answered along
with the recommendations, limitations and how these limitations could
become the basis of future research.\selectlanguage{american}%

