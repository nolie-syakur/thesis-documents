\selectlanguage{english}%

\chapter{Discussion}

At the beginning of our research, we have stated two research questions
that need to be answered. In this section, we will shortly discuss
our findings to answer our research questions.


\section{Research questions}

\ 

\textit{What are the characteristics of reported phishing emails?}

\ 

In \autoref{chap:research-questions}, we defined seven parameters
to characterize the phishing emails in our dataset. Based on our findings,
we can conclude in the following points:
\begin{itemize}
\item Based on \autoref{tab:Attachment(s)-analysis}, when attachment(s)
are included in a phishing email, it is likely to attach ZIP or HTML
file.
\item Requesting to click a URL(s) is the most prevalent instruction in
phishing emails. \autoref{tab:Methods-analysis} illustrates this
finding.
\item As we illustrated in \autoref{tab:Content-analysis}, most of the
phishing emails are using HTML code and provide URL(s). 
\item As the \autoref{tab:Target-analysis} shows, the financial sector
is the most common target.
\item \autoref{tab:Reason-classification} depicts that most of the phishing
emails use an account related concerns as a pretext.
\item Based on our finding from \autoref{tab:Persuasion-principle-analysis},
the authority principle is the most used persuasion technique in phishing
emails.
\item As we illustrated in \autoref{tab:Account-related-reason-1}, a phishing
email which has an account related concerns as a pretext, is likely
to include URL(s).
\item It can be seen from \autoref{tab:URL-presence-and-1} and \autoref{tab:Includes-attachment-and}
that phishers provide clear instructions on how recipients are meant
to act; phishing emails which include attachment(s) are likely to
request to open it and phishing emails which provide URL(s) are likely
to request to click on it. 
\item Based on our finding in \autoref{tab:URL-presence-and}, the URL(s)
in a phishing email are most likely different from the actual destination.
\item \autoref{tab:Document-related-reason} suggests that a government
targeted phish is likely to have a document related reason.
\item As we illustrated in \autoref{tab:Document-related-reason-1}, it
suggests that phishing emails which have document related reason as
a pretext, are likely to include attachment(s)
\end{itemize}
Moreover, it is worth pointing out that our finding on the detailed
financial sector in \autoref{fig:Detailed-of-financial} indicates
many of them are non-Dutch based financial institutions such as Bank
of America, Barclays Bank, and Llyods Bank. However, we found this
is not ideal because our dataset is coming from Dutch based organization.
Perhaps, this is because we conducted the analysis only on 207 unique
phishing emails in English language which is 2.45\% of the total reported
emails. We explain why we can only analyze 207 unique phishing emails
in English language in \autoref{sec:Limitation}. Despite of this
limitation, we emphasize that our current study portrays as a precursor
to a larger study of persuasion techniques and phishing emails in
general. By this reasoning, our method or measure instruments such
as algorithms, flowcharts, and variables would not be biasing the
method to what we may find in the reported phishing emails in Dutch
language.

\ 

\textit{To what extent are the persuasion principles used in phishing
emails?}

\ 

To answer the second research question, we look at the relationships
between the persuasion principles and the generic properties. With
this in mind, we have established 10 hypotheses related to these relationships
and we look whether the findings are consistent with these hypotheses.
\autoref{tab:Overview-of-verified} summarizes the overview of verified
hypotheses. Because almost all phishing emails use authority principle,
therefore, this implies all phishing email properties related to authority
principle are resulted in no significant relationship. 

When we look at financial targeted sector and scarcity principle,
we find that both financial and non financial targeted emails are
less chance to have scarcity principle. Apart from our hypothesis
related to financial sector and scarcity principle, if we look deeper,
administrator targeted emails are likely to have scarcity principle.
In contrary, non administrator targeted emails are less likely to
have scarcity principle. However, our finding suggests that the strength
of association between administrator targeted emails and scarcity
principle is weak.

The next finding on the relationship between e-commerce/retails targeted
emails, it indicates that this sector contributes less number of likeability
principle as both e-commerce/retails and non-ecommerce/retails targeted
emails have high number of non-likeability principle. Similarly, our
data suggests that there are no significant association between social
media targeted emails and social proof principle.

Another observation whether likeability and consistency have a relationship,
our data suggests that there is a significant association between
them. Our result signifies that the higher likeability, the lower
chance to have consistency principle. This support our hypothesis
which says the occurrence of likeability will impact the occurrence
of consistency. However, we find the strength of association between
the variables is weak.

When we look at account related phishing and scarcity, we find that
there is a highly significant relationship between them. This means
that if a phishing email uses an account related as a reason, it will
likely to use scarcity principle as a persuasion technique. Moreover,
the result suggests that account related reason and scarcity principle
have a strong relationship. 

Lastly, we find that there is a significant association between the
use of HTML and likeability principle. This suggests that likeability
phishing emails tend to use HTML code to persuade unsuspecting victim.
However, their strength of association is weak.

\begin{minipage}[t]{1\columnwidth}%
\begin{center}
\begin{longtable}{cccc}
\caption{\label{tab:Overview-of-verified}Overview of verified hypotheses}
\tabularnewline
\toprule 
Hypotheses & Category & Accept & Reject\tabularnewline
\midrule
\midrule 
H1 & A & \selectlanguage{american}%
\selectlanguage{american}%
 & X\tabularnewline
\midrule 
H2 & A & \selectlanguage{american}%
\selectlanguage{american}%
 & X\tabularnewline
\midrule 
H3 & A & \selectlanguage{american}%
\selectlanguage{american}%
 & X\tabularnewline
\midrule 
H4 & A & \selectlanguage{american}%
\selectlanguage{american}%
 & X\tabularnewline
\midrule 
H5 & A & \selectlanguage{american}%
\selectlanguage{american}%
 & X\tabularnewline
\midrule 
H6 & A & \selectlanguage{american}%
\selectlanguage{american}%
 & X\tabularnewline
\midrule 
H7 & A & X & \selectlanguage{american}%
\selectlanguage{american}%
\tabularnewline
\midrule 
H8 & B & X & \selectlanguage{american}%
\selectlanguage{american}%
\tabularnewline
\midrule 
H9 & B & X & \selectlanguage{american}%
\selectlanguage{american}%
\tabularnewline
\midrule 
H10 & B & X & \selectlanguage{american}%
\selectlanguage{american}%
\tabularnewline
\midrule 
H11 & A & \selectlanguage{american}%
\selectlanguage{american}%
 & X\tabularnewline
\midrule 
H12 & A & X & \selectlanguage{american}%
\selectlanguage{american}%
\tabularnewline
\midrule 
H13 & B & X & \selectlanguage{american}%
\selectlanguage{american}%
\tabularnewline
\midrule 
H14 & B & X & \selectlanguage{american}%
\selectlanguage{american}%
\tabularnewline
\midrule 
H15 & B & X & \selectlanguage{american}%
\selectlanguage{american}%
\tabularnewline
\midrule 
H16 & A & X & \selectlanguage{american}%
\selectlanguage{american}%
\tabularnewline
\end{longtable}
\par\end{center}%
\end{minipage}

{\footnotesize{}A = Related to persuasion principles}{\footnotesize \par}

{\footnotesize{}B = Related to generic structural properties}{\footnotesize \par}

\  

Overall, seven hypotheses in respect of persuasion principles are
rejected and three of them are accepted. Based on this result and
supported by our analysis on the relationship between persuasion principles
and target types in \autoref{sub:Relationship-between-persuasion}
and the relationship between persuasion principles and reason types
in \autoref{sub:Relationship-between-persuasion-1}, we can answer
our second research question by the following underlying perspectives: 
\begin{itemize}
\item The extensive use of authority as a persuasion technique in phishing
emails as opposed to social proof technique. However, our analysis
suggests that while the percentages are still high, authority principle
is less likely to be used in individual target type and social reason
type. (see \autoref{tab:Persuasion-principle-analysis}, \autoref{tab:Persuasion-principles-vs}
and \autoref{tab:Persuasion-principles-vs-1})
\item Depending on the target types and the reason types, three persuasion
principles (scarcity, consistency and likeability) are the next most
popular principles used in our dataset. (see \autoref{sub:Relationship-between-persuasion}
and \autoref{sub:Relationship-between-persuasion-1})
\item Scarcity principle will likely to be used when phishing emails are
coming from administrator target type and account reason type. (see
\autoref{tab:Chi-square-persuasion-principles} and \autoref{tab:Chi-square-tests-Persuasion})
\item Likeability principle affects the usage of HTML-based email and consistency
principle. (see \autoref{tab:use-HTML-and} and \autoref{tab:Likeability-and-consistency})
\end{itemize}

\section{Conclusion}

Our research was aimed at understanding how phishing emails use persuasion
techniques. The analysis consists of finding relationships between
persuasion techniques and generic properties of phishing emails.

The finding may be influenced by the fact that only one person (the
author) has coded the emails. Although, we had made a flowchart in
\autoref{fig:Flowchart-cialdini} to model our decisions in terms
of data coding, we believe persuasion techniques are personal and
difficult to find a consensual decision.

Nevertheless, by using parameters and hypotheses in \autoref{chap:research-questions},
we have been able to find the characteristics of phishing emails based
on persuasion techniques. Our approach has shown to be useful in identifying
critical characteristics and relationships between generic properties
of phishing emails and persuasion techniques. Three important findings
of our research are that (1) authority is the most popular persuasion
technique regardless of the target and the reason used, (2) depending
on the target types and the reason types, the next most popular persuasion
principles are scarcity, consistency and likeability and (3) scarcity
principle has a high involvement with administrator target type and
account related concerns. When we relate between target types and
the reason used in phishing emails, our suggestions for preventing
phishing can be described in the following points:
\begin{itemize}
\item If we assume that most people are more likely to comply with authority,
we suggest a legitimate institution such as financial sector should
never use emails as a medium of communication with its customers.
Instead, a legitimate institution should have its own secure messaging
system to communicate with its customers. This may reduce the risk
of costumers believing that phishing emails are real.
\item Even if a legitimate institution uses emails, they may use a simple
email validation system such as Sender Policy Framework (SPF), that
is designed to verify sender's email server before delivering all
legitimate email to the intended recipients. This can prevent a spoofed
email being delivered to the intended victim. 
\item Legitimate institution such as a bank could use of what its customers
have, such as phone number registered in the system or a token given
by the bank to the customers. A secret code in the email sent by the
bank should match the code delivered to the customer's phone or token.
This provides two factor authentication and will make more effort
for the phishers to spoof the email from the bank. 
\item It might be useful if security experts can create a library that contains
the most common words or phrases which signify authority and scarcity
principles, so that incoming email could be filtered using the library.
Our flowchart can be useful to help the development of the library
so that the conventional phishing email detection could be improved.
\item Our finding suggests that account related is the most used reason
in phishing emails, therefore, we suggest that anti phishing filtering
system should also focus to detect account related emails to prevent
them from being delivered to the intended victim.
\item Persuasion awareness in phishing emails is needed to help the end
users as the recipients to think before they respond to an email and
therefore, it could enhance users' ability to identify phishing emails.
\end{itemize}
Overall, the reflection from this research is that the phishers are
not only utilizing a technical strategy to trick the unsuspecting
victims, but also persuasion principles to obtain positive responses
from the victims. Our research exhibits an important aspect of phishing
emails so that future phishing email countermeasures should not only
be developed from a technical perspective but they should also be
able to resist from persuasion misuse. Continued research on persuasion
techniques in phishing emails is required to stay ahead of the phishers,
our method being a solid starting point in a real world analysis to
identify the underlying issue in phishing emails.


\section{\label{sec:Limitation}Limitation}

Although the research produces conclusive results, our findings need
to be assimilated in the backdrop of some limitations which arise
due to the complex nature of our methodology and research environment.
It is important for us to explain these limitations so that the readers
can understand the findings of our research in the proper context.
The first limitation is that we only get the data from one organization.
Our study is totally dependent on the information documented by this
organization, in which we do not know whether the sample data represents
national-wide or represents a certain area or criterion. The second
limitation is that, sometimes the email does not show the complete
structures because the reporter forwarded a suspected phishing email
as an attachment which removes essential element of it such as original
attachment(s) included in the original email. This causes our study
to be dependent on the reporter that reports to this organization
as well. The third limitation is language barrier. Few numbers of
suspected English-based phishing emails forwarded by the reporter
along an information in Dutch language. It might be useful to know
to understand the information provided by the reporter. Without understanding
of non-English language, we could also analyze the structural properties
of phishing emails such as whether the email use HTML or whether it
contain hidden URL. However, since our analysis was aimed to analyze
persuasion techniques in phishing emails, we need to have the language
proficiency to know which persuasion techniques were used. The fourth
limitation is that, our data classification is done by one person.
It means the coding of the data into associated variables could be
inaccurate. While the data coding to the generic structural properties
of phishing email could be justified, however, the data coding into
the persuasion principles could be an issue in terms of accuracy.
For example, one person claimed an email is attractive while another
person claimed it is not. This introduces the greatest limitation
to our research because it significantly impacts our results. The
fifth limitation is being unique dataset of the reported phishing
emails (440 phishing emails reduced to 207 unique phishing emails).
This resulted in smaller sample size of data and therefore, gives
a challenging task to find associations using Pearson chi-square method.


\section{\label{sec:Future-work}Future work}

Despite our considerable limitations, a recommendation as a follow-up
study to analyze phishing emails with Dutch language which we did
not observe, would be extremely desirable in order to test our findings.
We feel that any future research along this line will find our work
to be a useful starting point. Furthermore, we also recommend adding
resilient validation in data classification in terms of persuasion
principles by involving several people to have an objective decision.
It is also interesting to identify authority principle in regular
emails to make an objective perspective and compare with phishing
emails. By looking at our findings, future study regarding the success
of authority principle in phishing emails could be useful if we have
data of phishing emails which already has claimed a victim. Thus,
the goal for future research would be the success rate of authority
principle in phishing emails. In conjunction, controlled environment
to test persuasion awareness could be helpful to see whether it reduces
phishing victimization through emails or not. Lastly, as we understand
that persuasion principles in phishing email have some influence in
user's decisions, therefore, it is interesting if the future research
can build a simple game in terms of persuasion awareness to grab user's
attention to make the right decision. For instance, the flowchart
in \autoref{fig:Flowchart-cialdini}, can be adapted to ``snakes
and ladders'' game to alert users of the presence of persuasion principles
in an email they received.\selectlanguage{american}%

