\selectlanguage{english}%

\chapter{Discussion}

At the beginning of our research, we stated two research questions
that need to be answered. In this section, we discuss our findings
to answer these research questions.


\section{Research questions}

\ 

\textit{What are the characteristics of reported phishing emails?}

\ 

In \autoref{chap:research-questions}, we defined seven parameters
to characterize the phishing emails in our dataset. Based on our findings,
we can conclude the following points:
\begin{itemize}
\item Based on \autoref{tab:Attachment(s)-analysis}, when attachment(s)
are included in a phishing email, they are likely to be ZIP or HTML
files.
\item Requesting to click a URL(s) is the most prevalent instruction in
phishing emails. \autoref{tab:Methods-analysis} illustrates this
finding.
\item As we illustrated in \autoref{tab:Content-analysis}, most of the
phishing emails use HTML code and provide URL(s). 
\item As \autoref{tab:Target-analysis} shows, the financial sector is the
most common target.
\item \autoref{tab:Reason-classification} depicts that most of the phishing
emails use account-related concerns as a pretext.
\item Based on our finding from \autoref{tab:Persuasion-principle-analysis},
the authority principle is the most-used persuasion technique in phishing
emails.
\item As we illustrated in \autoref{tab:Account-related-reason-1}, a phishing
email that has an account-related concern as a pretext is likely to
include URL(s).
\item It can be seen from \autoref{tab:URL-presence-and-1} and \autoref{tab:Includes-attachment-and}
that phishers provide clear instructions on how recipients are meant
to act; phishing emails that include attachment(s) are likely to include
a request to open it and phishing emails which provide URL(s) are
likely to request to click on it. 
\item Based on our finding in \autoref{tab:URL-presence-and}, the URL(s)
in a phishing email are most likely different from the actual destination.
\item \autoref{tab:Document-related-reason} suggests that a government-targeted
phish is likely to have a document-related reason.
\item As we illustrated in \autoref{tab:Document-related-reason-1}, phishing
emails that have document-related reasons as a pretext are likely
to include attachment(s).
\end{itemize}
Moreover, it is worth pointing out that our finding on the detailed
financial sector in \autoref{fig:Detailed-of-financial} indicates
many of them are non-Dutch based financial institutions such as Bank
of America, Barclays Bank, and Lloyds Bank. However, we found this
is not ideal because our dataset came from a Dutch-based organization.
Perhaps this is because we conducted the analysis only on 207 unique
phishing emails in the English language, which is 2.45\% of the total
reported emails. We explain why we can only analyze 207 unique phishing
emails in the English language in \autoref{sec:Limitation}. Despite
this limitation, we emphasize that our current study can be seen as
a precursor to a larger study of persuasion techniques and phishing
emails in general. By this reasoning, our method or measure instruments
such as algorithms, flowcharts, and variables would not be biasing
the method to what we may find in the reported phishing emails in
the Dutch language.

\ 

\textit{To what extent are the persuasion principles used in phishing
emails?}

\ 

To answer the second research question, we look at the relationships
between the persuasion principles and the generic properties. With
this in mind, we have established 10 hypotheses related to these relationships
and we look whether the findings are consistent with these hypotheses.
\autoref{tab:Overview-of-verified} summarizes the overview of verified
hypotheses. Because almost all phishing emails use the authority principle,
this implies all phishing email properties related to the authority
principle resulted in no significant relationship. 

When we look at the targeted sector and scarcity principle, we find
that both financial and non-financial targeted emails are less chance
to have scarcity principle. Apart from our hypothesis related to the
financial sector and scarcity principle, we see that administrator-targeted
emails are likely to have scarcity principle. In contrast, non-administrator
targeted emails are less likely to have scarcity principle. However,
our findings suggest that the strength of association between administrator-targeted
emails and the scarcity principle is weak. 

The next finding on the relationship between e-commerce/retail sector
targeted emails indicates that this sector less employs the likeability
principle as both e-commerce/retail sector and non- e-commerce/retail
sector targeted emails have a high number featuring the non-likeability
principle. Similarly, our data suggests that there are no significant
association be-tween social media targeted emails and the social proof
principle.

Our data suggests that there is a significant association between
likeability and consistency. According to our findings, the higher
the likeability, the lower the chance of featuring the consistency
principle. This support our hypothesis that says that the occurrence
of likeability will impact the occurrence of consistency. However,
we find the strength of association between the variables is weak.

When we look at account-related phishing and scarcity, we find that
there is a highly significant relationship between them. This means
that if a phishing email uses an account-related reason, it will likely
use the scarcity principle as a persuasion technique. Moreover, the
result suggests that account-related reasons and the scarcity principle
have a strong relationship.

Lastly, we find that there is a significant association between the
use of HTML and the likeability principle. This suggests that likeability
phishing emails tend to use HTML code to persuade unsuspecting victims.
However, their strength of association is weak.

\begin{minipage}[t]{1\columnwidth}%
\begin{center}
\begin{longtable}{cccc}
\caption{\label{tab:Overview-of-verified}Overview of verified hypotheses}
\tabularnewline
\toprule 
Hypotheses & Category & Accept & Reject\tabularnewline
\midrule
\midrule 
H1 & A & \selectlanguage{american}%
\selectlanguage{american}%
 & X\tabularnewline
\midrule 
H2 & A & \selectlanguage{american}%
\selectlanguage{american}%
 & X\tabularnewline
\midrule 
H3 & A & \selectlanguage{american}%
\selectlanguage{american}%
 & X\tabularnewline
\midrule 
H4 & A & \selectlanguage{american}%
\selectlanguage{american}%
 & X\tabularnewline
\midrule 
H5 & A & \selectlanguage{american}%
\selectlanguage{american}%
 & X\tabularnewline
\midrule 
H6 & A & \selectlanguage{american}%
\selectlanguage{american}%
 & X\tabularnewline
\midrule 
H7 & A & X & \selectlanguage{american}%
\selectlanguage{american}%
\tabularnewline
\midrule 
H8 & B & X & \selectlanguage{american}%
\selectlanguage{american}%
\tabularnewline
\midrule 
H9 & B & X & \selectlanguage{american}%
\selectlanguage{american}%
\tabularnewline
\midrule 
H10 & B & X & \selectlanguage{american}%
\selectlanguage{american}%
\tabularnewline
\midrule 
H11 & A & \selectlanguage{american}%
\selectlanguage{american}%
 & X\tabularnewline
\midrule 
H12 & A & X & \selectlanguage{american}%
\selectlanguage{american}%
\tabularnewline
\midrule 
H13 & B & X & \selectlanguage{american}%
\selectlanguage{american}%
\tabularnewline
\midrule 
H14 & B & X & \selectlanguage{american}%
\selectlanguage{american}%
\tabularnewline
\midrule 
H15 & B & X & \selectlanguage{american}%
\selectlanguage{american}%
\tabularnewline
\midrule 
H16 & A & X & \selectlanguage{american}%
\selectlanguage{american}%
\tabularnewline
\end{longtable}
\par\end{center}%
\end{minipage}

{\footnotesize{}A = Related to persuasion principles}{\footnotesize \par}

{\footnotesize{}B = Related to generic structural properties}{\footnotesize \par}

\  

Overall, seven hypotheses in respect of persuasion principles are
rejected and three of them are accepted. Based on this result and
supported by our analysis of the relationship between persuasion principles
and target types in \autoref{sub:Relationship-between-persuasion}
and the relationship between persuasion principles and reason types
in \autoref{sub:Relationship-between-persuasion-1}, we can answer
our second research question with the following underlying perspectives:
\begin{itemize}
\item The extensive use of authority as a persuasion technique in phishing
emails as opposed to social proof technique. However, our analysis
suggests that while the percentages are still high, authority principle
is less likely to be used in individual target type and social reason
type (see \autoref{tab:Persuasion-principle-analysis}, \autoref{tab:Persuasion-principles-vs}
and \autoref{tab:Persuasion-principles-vs-1}).
\item Depending on the target types and the reason types, three persuasion
principles -- scarcity, consistency and likeability -- are the next
most popular principles used in our dataset (see \autoref{sub:Relationship-between-persuasion}
and \autoref{sub:Relationship-between-persuasion-1}).
\item The scarcity principle will likely be used when phishing emails come
from the administrator target type and account reason type (see \autoref{tab:Chi-square-persuasion-principles}
and \autoref{tab:Chi-square-tests-Persuasion}).
\item The likeability principle affects the usage of HTML-based email and
consistency principle (see \autoref{tab:use-HTML-and} and \autoref{tab:Likeability-and-consistency}).
\end{itemize}

\section{Conclusion}

Our research was aimed at understanding how phishing emails use persuasion
techniques. The analysis consists of finding relationships between
persuasion techniques and generic properties of phishing emails.

The findings may be influenced by the fact that only one person (the
author) has coded the emails. Although we made a flowchart in \autoref{fig:Flowchart-cialdini}
to model our decisions in terms of data coding, we believe persuasion
techniques are personal and difficult to find a consensual decision.

Nevertheless, by using parameters and hypotheses in \autoref{chap:research-questions},
we have been able to find the characteristics of phishing emails based
on persuasion techniques. Our approach has proven useful in identifying
critical characteristics and relationships between generic properties
of phishing emails and persuasion techniques. Three important findings
of our research are that: (1) authority is the most popular persuasion
technique regardless of the target and the reason used; (2) depending
on the target types and the reason types, the next most popular persuasion
principles are scarcity, consistency and likeability; and (3) scarcity
principle has a high involvement with administrator target type and
account-related concerns. 

When we relate between target types and the reason used in phishing
emails, our suggestions for preventing phishing can be described in
the following points:
\begin{itemize}
\item If we assume that most people are more likely to comply with authority,
we suggest a legitimate institution should never use emails as a medium
of communication with its customers. Instead, a legitimate institution
should have its own secure messaging system to communicate with its
customers. This may reduce the risk of costumers believing that phishing
emails are real.
\item Even if a legitimate institution uses emails, they may use a simple
email validation system such as Sender Policy Framework (SPF), which
is designed to verify the sender\textquoteright s email server before
delivering all legitimate email to the intended recipients. This can
prevent a spoofed email being delivered to the intended victim.
\item Legitimate institutions such as banks could use what its customers
have, such as a phone number registered in the system or a token given
by the bank to the customers. A secret code in the email sent by the
bank should match the code delivered to the customer\textquoteright s
phone or token. This would provide a two-factor authentication and
would make it more difficult for phishers to spoof bank emails.
\item It might be useful if security experts can create a library that contains
the most common words or phrases that signify authority and scarcity
principles, so that incoming email could be filtered using the library.
Our flowchart can be useful to help the development of the library
so that the conventional phishing email detection can be improved.
\item Our findings suggest that account-related is the most used reason
in phishing emails. Therefore, we suggest that anti phishing filtering
systems should also focus on detecting account-related emails to prevent
them from being delivered to the intended victim.
\item Persuasion awareness in phishing emails is needed to help the end
users think before they respond to an email, and to enhance users'
ability to identify phishing emails.
\end{itemize}
Overall, the reflection from this research is that the phishers are
not only utilizing a technical strategy to trick the unsuspecting
victims, but also persuasion principles to obtain positive responses
from the victims. Our research exhibits an important aspect of phishing
emails so that future phishing email countermeasures should not only
be developed from a technical perspective but they should also be
able to resist from persuasion misuse. Continued research on persuasion
techniques in phishing emails is required to stay ahead of the phishers.
Our method is a solid starting point in a real world analysis to identify
the underlying issue in phishing emails.


\section{\label{sec:Limitation}Limitation}

Although the research produced conclusive results, our findings need
to be assimilated in the backdrop of some limitations that arose due
to the complex nature of our methodology and the research environment.
It is important for us to explain these limitations so that the readers
can understand the findings of our research in the proper context. 

The first limitation is that we got the data from only one organization.
Our study is totally dependent on the information documented by this
organization, and we do not know whether the sample data represents
the Netherlands overall or represents a certain area or criterion. 

The second limitation is that sometimes the emails did not show the
complete structures because the reporter forwarded a suspected phishing
email as an attachment, which removes essential elements of it such
as any attachment(s) included in the original email. This causes our
study to be dependent on the reporter that reports to this organization
as well. 

The third limitation is the language barrier. Few of the English-based
phishing emails forwarded by the reporter had information in Dutch.
It might be useful to know to understand the information provided
by the reporter. Without understanding non-English language, we could
also analyze the structural properties of phishing emails, such as
whether the email uses HTML or whether it contain hidden URL. However,
since our analysis was aimed at analyzing persuasion techniques in
phishing emails, we need to have the language proficiency to know
which persuasion techniques were used. 

The fourth limitation is that our data classification was done by
one person. This means the coding of the data into associated variables
could be inaccurate. While the data coding to the generic structural
properties of phishing email could be justified, however, the data
coding into the persuasion principles could be an issue in terms of
accuracy. For example, one person can claim an email is attractive
while another person can claim it is not. This introduces the greatest
limitation to our research because it significantly impacts our results. 

The fifth limitation is the unique dataset of the reported phishing
emails (440 phishing emails reduced to 207 unique phishing emails).
This resulted in a smaller sample size of data and therefore it was
a challenging task to find associations using Pearson chi-square method.


\section{\label{sec:Future-work}Future work}

A follow-up study to analyze phishing emails with Dutch language,
which we did not observe, would be extremely desirable in order to
test our findings. We feel that any future research along this line
will find our work to be a useful starting point. Furthermore, we
also recommend adding resilient validation in data classification
in terms of persuasion principles by involving several people to have
an objective decision. It is also interesting to identify the authority
principle in regular emails to make an objective perspective and compare
with phishing emails. By looking at our findings, future study regarding
the success of authority principle in phishing emails could be useful
if we have data of phishing emails that have already claimed a victim.
Thus, the goal for future research would be the success rate of the
authority principle in phishing emails. In conjunction, a controlled
environment to test persuasion awareness would be helpful to see whether
it reduces phishing victimization through emails or not. Finally,
as we understand that persuasion principles in phishing email have
some influence in user\textquoteright s decisions, it would be interesting
if future research can build a simple game in terms of persuasion
awareness to grab a user\textquoteright s attention to make the right
decision. For instance, the flowchart in \autoref{fig:Flowchart-cialdini}
can be adapted to a \textquotedblleft snakes and ladders\textquotedblright{}
game to alert users of the presence of persuasion principles in an
email they receive.\selectlanguage{american}%

