\selectlanguage{english}%

\chapter{\label{chap:research-questions}Research Questions and Hypotheses}

This chapter addresses the rationale of our main research questions
and hypotheses to meet our research goal. We aim to answer these research
questions by the analysis of data collected from a security organization
based in Netherlands. First off, we wanted to know the characteristics
of phishing email based on structural properties in our corpus.

\rule[0.5ex]{1\columnwidth}{1pt}

\textit{RQ1: What are the characteristics of the phishing emails?}

\rule[0.5ex]{1\columnwidth}{1pt}

The characteristics of phishing emails in our dataset are determined
by the following parameters:
\begin{itemize}
\item How often phishing email include an attachment(s) and what specific
attachment is the most frequent.
\item Prevalent instructions
\item Content characteristics
\item The most targeted institutions
\item The reasons that are frequently being used
\item Persuasion principles characteristics
\item Relationship between generic properties
\end{itemize}
To find out these characteristics, variable establishment of structural
properties will be addressed in \autoref{sub:variables}.

Secondly, we wanted to know to what extent the involvement of persuasion
principles are used in phishing emails and how relevant are they to
the generic phishing email properties. 

\textit{\rule[0.5ex]{1\columnwidth}{1pt}}

\textit{RQ2: To what extent are the persuasion principles used in
phishing emails?}

\rule[0.5ex]{1\columnwidth}{1pt}

We established 16 hypotheses to indicate the relationship between
generic properties and relevancy of persuasion principle to these
properties. H8, H9, H10, H13, H14, H15 will partly answer RQ1 in respect
to the relationship between generic properties and the rest will answer
RQ2. We conduct an analysis of phishing emails based on Cialdini's
principles. In order to conduct the analysis, we established our decision
making to classify which persuasive elements that exist in a phishing
email. This process will be explained in \autoref{sub:cialdini}. 

In our coding of cialdini's principles and phishing email dataset,
we identified phishing emails with fake logos and signatures that
may mistakenly regard them as legitimate by average internet users.
For example in the context of phishing email, signature such as ``Copyright
2013 PayPal, Inc. All rights reserved'' or ``Administrator Team''
and Amazon logo were used to show the ``aura of legitimacy''. In
the real world society, telemarketers and seller has been using authoritative
element to increase the chance of potential consumer's compliance
\cite{telemarket:2013}. It means that they have to provide information
in a confident way. Consumers will have their doubt if sellers unsure
and nervous when they offer their product and services to consumers.
This principle has been one of the strategies in a social engineering
attack to acquire action and response from a target \cite{npdn:2013}.

It is makes sense if government has the authority to compose laws
and regulations and to control its citizens. Government sector includes
court and police department also authorize to execute penalties if
any wrongdoing happens within their jurisdiction. However, government
may not have to be likeable to enforce their rules and regulation.
Similarly, an administrator who control his network environment may
behave in a similar fashion as government. Hence, in our dataset we
hypothesize that

\rule[0.5ex]{1\columnwidth}{1pt}

\textit{H1: There will be a significant association between Government
sector and authority principle}

\textit{H2: Phishing emails which targeting Administrator will likely
to have authority principle }

\rule[0.5ex]{1\columnwidth}{1pt}

Similar to authority principle that may trigger reactance, scarce
items and shortage may produce immediate compliance from people. In
essence, people will react when their freedom is restricted about
valuable matter when they think they are capable to make a choice
among different options \cite{pennebaker1976american}. For example
in phishing email context, an email from Royal Bank inform us that
we have not been logged into our online banking account for a quite
some time, as a security measure, they must suspend our online account
and if we would like to continue to use the online banking facility,
we have been asked to click the URL provided. Potential victim may
perceives their online banking account as their valuable matter to
access facility and information about their savings. Consequently,
potential vicim may react to the request because of their account
could be scarce and restricted. In the real world example, a hard
worker bank customer who perceives money is a scarce item may immediately
react when his bank inform him that he is in danger of losing his
savings due to ``security breach''. We therefore hypothesize that

\rule[0.5ex]{1\columnwidth}{1pt}

\textit{H3: There will be a significant correlation between Financial
sector and scarcity principle}

\rule[0.5ex]{1\columnwidth}{1pt}

As we describe in our decision making consideration section, people
tend to trust those they like. In a context of persuasion, perpetrators
may find it more difficult to portray physical attractiveness, instead
they are relying on emails, websites and phone calling \cite{dotterweich2006practicality}.
To exhibit charm or charisma to the potential victims, perpetrators
may gain their trust by establishing friendly emails, affectionate
websites and soothing voice over the phone. In the phishing email
context, Amazon praises our existence in an appealing fashion and
extremely values our account security so that no one can break it.
Based on this scenario, E-commerce/Retails sector may applied likeability
principles to gain potential customers. We therefore hypothesize that

\rule[0.5ex]{1\columnwidth}{1pt}

\textit{H4: Phishing emails which targeting E-Commerce/Retails will
likely to have a significant relationship with likeability principle}

\rule[0.5ex]{1\columnwidth}{1pt}

Tajfel, et al. argued that people often form their own perception
based on their relationship with others in a certain social circles
\cite{tajfel2004social}. This lead to affection of something when
significant others have something to do with it. Social proof is one
of the social engineering attacks based on the behavioral modeling
and conformance \cite{workman:2008}. For example, we tend to comply
to a request when a social networking site asks us to visit a website
or recommends something and mention that others have visited the website
as well. Thus, we hypothesize that 

\rule[0.5ex]{1\columnwidth}{1pt}

\textit{H5: Phishing emails which targeting Social networks will likely
to have signification association with social proof principle }

\rule[0.5ex]{1\columnwidth}{1pt}

As we describe in our decision making consideration section, authority
has something to do with ``aura of legitimacy''. This principle
may lead to suggest the limitation on something that we deemed valuable.
For example, a perpetrator masquerades as an authority and dressed
as police officer halted us on the road, the perpetrator may tell
us that we did something wrong and he will held our driving license
if we do not pay him the fine. In the phishing email context, an email
masquerades as ``System Administrator'' may tell us that we exceeded
our mailbox quota, so the administrator must freeze our email account
and we could re-activate it by clicking the URL provided in the email.
Based on this scenario, we know that it has authority principle and
also has scarcity principle. Therefore, we hypothesize that

\rule[0.5ex]{1\columnwidth}{1pt}

\textit{H6: There will be a significant relationship between authority
principle and scarcity principle}

\rule[0.5ex]{1\columnwidth}{1pt}

We often stumbled a group of people requesting to donate some of our
money to the unfortunate people. Evidently, they would use physical
attractiveness and kind words to get our commitment to support those
people. Once they have got our commitment, they start asking for donation
and we tend to grant their request and give some of our money to show
that we are committed. Phishing email could be similar, for example,
Paypal appreciates our membership on their system and PayPal kindly
notifies us that in the membership term of agreement, they would performing
annual membership confirmation from its customers. Based on this scenario,
we know that the email has likeability principle and also has consistency
principle. We would like to know if it is the case with phishing email
in our dataset. Therefore, we hypothesize that

\rule[0.5ex]{1\columnwidth}{1pt}

\textit{H7: The occurrence of likeability in a phish will impact the
occurrence of consistency}

\rule[0.5ex]{1\columnwidth}{1pt}

We think it make sense if a fraudster tries to make his fake product
as genuine as possible and hide the fabricated element of his product.
There are also fraudster that did not make his product as identical
as the legitimate product. In the phishing email context, we perceives
fake product as URL in the email, phishers do not necessarily obfuscates
the real URL with something else. Logically, such phishers do not
aim to make a high quality of bogus email, rather they aim to take
chances in getting potential victims that are very careless. This
leads to our hypothesis that say

\rule[0.5ex]{1\columnwidth}{1pt}

\textit{H8: Phishing emails that include URL will likely to be obfuscated}

\rule[0.5ex]{1\columnwidth}{1pt}

It is conspicuous from our knowledge if a sales agent tries to sell
us a product, it would be followed by the request element to buy the
product as well. However, it will not make sense if he tries to sell
his product but he requests to buy another company's product. In other
words, if we have something to sell, we do not just display our product
without asking people's attention to look at our product. For example
in phishing email context, phishers may include URL or attachment
in the body of the email and also they may request unsuspecting victim
to click the URL or to open the attachment. This leads us to two hypotheses
which state

\rule[0.5ex]{1\columnwidth}{1pt}

\textit{H9: Phishing emails that include URL will likely to request
to click the URL}

\textit{H10: Phishing emails that include attachment will likely to
request to open the attachment}

\rule[0.5ex]{1\columnwidth}{1pt}

We sometimes find it suspicious if a person dressed as police officer
that does not have a badge carried with him, unless he is a fake police
officer. Consequently, a fake police officer may use a fake badge
to build up even more ``aura of legitimacy''. Evidently, Cialdini
suggests the increment of passerby who have stop and stare at the
sky by 350 percent with suit and tie instead of casual dress \cite{cialdini:2001}.
Hence, we correlate that a person who wears police uniform and a fake
badge in the real world context as authority principle and the presence
of image in the phishing mail context. Another example, an email that
masquerades Apple company, may clone Apple company logo or trademark
to its content to increase the chance of potential victim's response
or increase the |believability'' if you will. Thus, we hypothesize
that

\rule[0.5ex]{1\columnwidth}{1pt}

\textit{H11: Phishing emails that have authority principle will likely
to include an image to its content}

\rule[0.5ex]{1\columnwidth}{1pt}

Apart from the target analysis, we also investigate the reason why
potential victim responds to phisher's request. Phishing email that
implies our account expiration would have scarcity principle because
the account itself may very valuable for us and is in danger to be
expired or terminated. Therefore, we hypothesize that

\rule[0.5ex]{1\columnwidth}{1pt}

\textit{H12: There will be a significant association between account
related reason and scarcity principle }

\rule[0.5ex]{1\columnwidth}{1pt}

Similar from the hypothesis H12, it is sensible if a phishing email
which contains account related reason such as reset password or security
update, may tend to have a URL for the potential victim to be redirected
towards phisher's bogus website or malware. Regardless of the target,
based on our initial coding of the dataset we found that account related
reason in a phishing email needs an immediate action greater than
other reasons. Therefore, phishers may likely to include a URL to
have immediate response from the potential victim. This leads to our
hypothesis that say

\rule[0.5ex]{1\columnwidth}{1pt}

\textit{H13: Phishing emails which have account related reason will
likely to have URL }

\rule[0.5ex]{1\columnwidth}{1pt}

When a phishing email has document related reason such as review some
document reports or court notice, it may tend to impersonate government
to make the email sensible enough to persuade potential victim more
than other targets. We therefore hypothesize that

\rule[0.5ex]{1\columnwidth}{1pt}

\textit{H14: Phishing emails which targeting government sector will
likely to have document related reason }

\rule[0.5ex]{1\columnwidth}{1pt}

Analogous with the hypothesis \textit{H14}, it is make sense if a
phishing email which has document related reason such as reviewing
contract agreement or reviewing resolution case, would tend to have
a file to be attached. We therefore hypothesize that

\rule[0.5ex]{1\columnwidth}{1pt}

\textit{H15: Phishing emails which have document related reason will
likely to include attachment}

\rule[0.5ex]{1\columnwidth}{1pt}

We think it is make sense if a phishing email which use HTML to present
their email design may tend to increase the attractiveness to the
potential victim. Consequently, unsuspected victim may respond to
the request just because of the email design is attractive. Therefore,
we hypothesize that

\rule[0.5ex]{1\columnwidth}{1pt}

\textit{H16: Phishing emails which use HTML will have a significant
association with likeability principle}

\rule[0.5ex]{1\columnwidth}{1pt}\selectlanguage{american}%

