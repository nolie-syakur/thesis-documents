\selectlanguage{english}%

\chapter{\label{chap:research-questions}Research Questions and Hypotheses}

This chapter addresses the rationale of our main research questions
and hypotheses to meet our research goal. We aim to answer these research
questions with analysis of data collected by a security organization
based in Netherlands. First off, we wanted to know the characteristics
of phishing email based on structural properties in our corpus.

\rule[0.5ex]{1\columnwidth}{1pt}

\textit{RQ1: What are the characteristics of the phishing emails?}

\rule[0.5ex]{1\columnwidth}{1pt}

The characteristics of phishing emails in our dataset are determined
by the following parameters:
\begin{itemize}
\item How often phishing emails include an attachment(s) and what specific
attachment is the most frequent;
\item Prevalent instructions;
\item Content characteristics;
\item The most targeted institutions;
\item The reasons that are frequently being used;
\item Persuasion principles characteristics;
\item Relationship between generic properties.
\end{itemize}
To find out these characteristics, variable establishment of structural
properties will be addressed in \autoref{sub:variables}.

Secondly, we wanted to know to what extent the involvement of persuasion
principles are used in phishing emails and how relevant they are to
the generic phishing email properties.

\textit{\rule[0.5ex]{1\columnwidth}{1pt}}

\textit{RQ2: To what extent are persuasion principles used in phishing
emails?}

\rule[0.5ex]{1\columnwidth}{1pt}

We established 16 hypotheses to indicate the relationship between
generic properties and relevance of persuasion principle to these
properties. H8, H9, H10, H13, H14, H15 will answer RQ1 in respect
to the relationship between generic properties and the rest will answer
RQ2. We conducted an analysis of phishing emails based on Cialdini's
principles. In order to conduct the analysis, we established our decision
making to classify which persuasive elements that exist in a phishing
email. This process will be explained in \autoref{sub:cialdini}. 

In our coding of Cialdini\textquoteright s principles and phishing
email dataset, we identified phishing emails with fake logos and signatures
that may mistakenly be regarded as legitimate by average internet
users. For example, in the context of phishing email, signatures such
as \textquotedblleft Copyright 2013 PayPal, Inc. All rights reserved\textquotedblright{}
or \textquotedblleft Administrator Team\textquotedblright{} and the
Amazon logo were used to show the \textquotedblleft aura of legitimacy\textquotedblright .
In the real world, telemarketers and sellers use authoritative element
to increase the chance of potential consumers' compliance \cite{telemarket:2013}.
This means that they have to provide information in a confident way.
Consumers will have their doubts if sellers are unsure and nervous
when they offer their products and services. This principle has been
one of the strategies in a social engineering attack to acquire action
and response from a target \cite{npdn:2013}.

It makes sense if a government has the authority to compose laws and
regulations and to control its citizens. Government sectors including
court and police departments are also authorized to execute penalties
if any wrongdoing happens within their jurisdiction. However, a government
may not have to be likeable to enforce their rules and regulation.
An administrator who controls his/her network environment may behave
in a similar fashion to as government. Hence, in our dataset we hypothesize
that:

\rule[0.5ex]{1\columnwidth}{1pt}

\textit{H1: There will be a significant association between government
sector and authority principle}

\textit{H2: Phishing emails which targeting an administrator will
likely have authority principle }

\rule[0.5ex]{1\columnwidth}{1pt}

Similar to the authority principle that may trigger compliance, scarce
items and shortage may produce immediate compliance from people. In
essence, people will react when their freedom is restricted about
a valuable matter when they think they are capable of making a choice
among different options \cite{pennebaker1976american}. For example,
in the phishing email context, we may get an email from Royal Bank
informing us that we have not been logged into our online banking
account for quite some time, and as a security measure they must suspend
our online account. If we would like to continue to use the online
banking facility, we must click the URL provided. The potential victim
may perceive their online banking account as their valuable matter
to access facility and information about their savings. Consequently,
they may react to the request because their account could be scarce
and restricted. In a real world example, a hard-working bank customer
who perceives money is a scarce item may immediately react when their
bank informs them that they are in danger of losing their savings
due to a \textquotedblleft security breach\textquotedblright . We
therefore hypothesize that:

\rule[0.5ex]{1\columnwidth}{1pt}

\textit{H3: There will be a significant correlation between Financial
sector and scarcity principle}

\rule[0.5ex]{1\columnwidth}{1pt}

As we describe in our decision-making consideration section, people
tend to trust those they like. In a context of persuasion, perpetrators
may find it more difficult to portray physical attractiveness, as
they are relying on emails, websites and phone calling \cite{dotterweich2006practicality}.
To exhibit charm or charisma to the potential victims, perpetrators
may gain their trust by establishing friendly emails, affectionate
websites and soothing voices over the phone. In the phishing email
context, Amazon praises our existence in an appealing fashion and
extremely values our account security so that no one can break it.
Based on this scenario, the e-commerce/retail sector may apply likeability
principles to gain potential customers. We therefore hypothesize that:

\rule[0.5ex]{1\columnwidth}{1pt}

\textit{H4: Phishing emails that target e-commerce/retail companies
will likely have a significant relationship with likeability principle}

\rule[0.5ex]{1\columnwidth}{1pt}

Tajfel et al. argue that people often form their own perception based
on their relationship with others in certain social circles \cite{tajfel2004social}.
This leads to an affection of something when significant others have
something to do with it. Social proof is one of the social engineering
attacks based on the behavioral modeling and conformance \cite{workman:2008}.
For example, we tend to comply to a request when a social networking
site asks us to visit a website or recommends something and mention
that others have visited the website as well. Thus, we hypothesize
that:

\rule[0.5ex]{1\columnwidth}{1pt}

\textit{H5: Phishing emails that target social networks will likely
have significant association with social proof principle }

\rule[0.5ex]{1\columnwidth}{1pt}

As we describe in our decision-making consideration section, authority
has something to do with \textquotedblleft aura of legitimacy\textquotedblright .
This principle may lead to suggest the limitation on something that
we deem valuable. For example, if a perpetrator masquerading as an
authority and dressed as police officer stops us on the road, the
perpetrator may tell us that we did something wrong and that they
will take our driving license if we do not pay them the fine. In the
phishing email context, an email masquerading as \textquotedblleft System
Administrator\textquotedblright{} may tell us that we have exceeded
our mailbox quota, so the administrator must freeze our email account,
and that we can reactivate it by clicking the URL provided in the
email. This scenario uses both the authority principle and scarcity
principle. Therefore, we hypothesize that:

\rule[0.5ex]{1\columnwidth}{1pt}

\textit{H6: There will be a significant relationship between authority
principle and scarcity principle}

\rule[0.5ex]{1\columnwidth}{1pt}

We often stumble upon a group of people requesting us to donate some
of our money to more unfortunate people. Of course, they use physical
attractiveness and kind words to get our commitment to support those
people. Once they have got our commitment, they start asking for a
donation, and we tend to grant their request and give some of our
money to show that we are committed. Phishing emails can work in a
similar way. For example, an email may say that Paypal appreciates
our membership and kindly notifies us that in the membership term
of the agreement they must perform an annual membership confirmation
of its customers. Based on this scenario, we know that the email has
the likeability principle and consistency principle. We would like
to know if it is the case with phishing email in our dataset. Therefore,
we hypothesize that:

\rule[0.5ex]{1\columnwidth}{1pt}

\textit{H7: The occurrence of likeability in a phish will impact the
occurrence of consistency}

\rule[0.5ex]{1\columnwidth}{1pt}

We think it make sense if a fraudster tries to make their fake product
as genuine as possible and hide the fabricated element of their product.
There are also fraudsters that do not make their product identical
to the legitimate product. In the phishing email context, we perceive
fake products as URLs in an email. Phishers do not necessarily hide
the real URL with something else. Logically, such phishers do not
aim to make a high quality of bogus email. Rather they aim to take
chances in getting potential victims that are very careless. This
leads to our hypothesis that:

\rule[0.5ex]{1\columnwidth}{1pt}

\textit{H8: Phishing emails that include URLs will likely different
than the actual destination}

\rule[0.5ex]{1\columnwidth}{1pt}

We know from experience that if a sales agent tries to sell us a product,
it would be followed by the request element to buy the product as
well. However, it would not make sense if they try to sell their product
but requests to buy another company's product. In other words, if
we have something to sell, we do not just display our product without
asking for people's attention to look at our product. In the phishing
email context, phishers may include a URL or attachment in the body
of the email and they may also request the unsuspecting victim to
click the URL or to open the attachment. This leads us to the following
two hypotheses:

\rule[0.5ex]{1\columnwidth}{1pt}

\textit{H9: Phishing emails that include URLs will likely request
users to click on the URL}

\textit{H10: Phishing emails that include attachment will likely request
users to open the attachment}

\rule[0.5ex]{1\columnwidth}{1pt}

We sometimes find it suspicious if a person dressed as a police officer
does not have a badge. Consequently, a fake police officer may use
a fake badge to build up even more \textquotedblleft aura of legitimacy\textquotedblright .
Cialdini suggests the increment of passers-by who stop and stare at
the sky by 350 percent with a person in suit and tie instead of casual
dress \cite{cialdini:2001}. Hence, we correlate that a person who
wears police uniform and a fake badge in the real world context as
authority principle and the presence of an image in the phishing mail
context. Similarly, an email that masquerades as Apple may clone the
Apple company logo or trademark to its content to increase the chance
of a potential victim's response -- to increase the \textquotedblleft believability\textquotedblright .
Thus, we hypothesize that:

\rule[0.5ex]{1\columnwidth}{1pt}

\textit{H11: Phishing emails that have the authority principle will
likely include an image in its content}

\rule[0.5ex]{1\columnwidth}{1pt}

Apart from the target analysis, we also investigate the reason why
potential victims respond to phishers' requests. Phishing emails that
imply account expiration incorporate the scarcity principle because
the account itself may be very valuable for us and we fear it expiring
or being terminated. Therefore, we hypothesize that:

\rule[0.5ex]{1\columnwidth}{1pt}

\textit{H12: There will be a significant association between account-related
reasons and scarcity principle }

\rule[0.5ex]{1\columnwidth}{1pt}

Similar to hypothesis H12, it is likely that a phishing email that
contains account-related reasons such as reset password or security
update will have a URL for the potential victim to be redirected towards
the phisher's bogus website or malware. Regardless of the target,
based on our initial coding of the dataset we found that account-related
reasons in a phishing email requires more immediate action than other
reasons. Therefore, phishers may likely to include a URL to have an
immediate response from the potential victim. This leads to our hypothesis
that:

\rule[0.5ex]{1\columnwidth}{1pt}

\textit{H13: Phishing emails that have account-related reasons will
likely include URL(s) }

\rule[0.5ex]{1\columnwidth}{1pt}

When a phishing email has document-related reasons such as reviewing
some document reports or court notice, they tend to impersonate a
government to make the email realistic enough to persuade the potential
victim more than other targets. We therefore hypothesize that:

\rule[0.5ex]{1\columnwidth}{1pt}

\textit{H14: Phishing emails which targeting government sector will
likely have document-related reasons}

\rule[0.5ex]{1\columnwidth}{1pt}

Analogous with hypothesis H14, it make sense that a phishing email
that has a document-related reason such as reviewing contract agreement
or reviewing resolution case, would tend to have a file attached.
We therefore hypothesize that:

\rule[0.5ex]{1\columnwidth}{1pt}

\textit{H15: Phishing emails which have document related reason will
likely to include attachment}

\rule[0.5ex]{1\columnwidth}{1pt}

We think it makes sense if a phishing email that uses HTML to present
their email design to be more attractive to the potential victim.
Consequently, an unsuspecting victim may respond to the request just
because the email design is attractive. Therefore, we hypothesize
that:

\rule[0.5ex]{1\columnwidth}{1pt}

\textit{H16: Phishing emails which use HTML will have a significant
association with likeability principle}

\rule[0.5ex]{1\columnwidth}{1pt}\selectlanguage{american}%

