\begingroup
\let\clearpage\relax
\let\cleardoublepage\relax

\pdfbookmark[1]{Abstract}{Abstract}


\chapter*{Abstract}

As the barrier to abuse system vulnerabilities has been raised significantly
with time, attacking users' psyche has rapidly become a more efficient
and effective alternative. The usage of email as a media electronic
of communication has been exploited by phishers to deliver their attacks.
The success of a phishing attack through distributed emails is determined
by the response from the unsuspecting victims. Although persuasion
can be used as a tool for a good reason, they can also be used for
a malicious reason by the phishers to get a positive response from
an intended victim in phishing emails.

To protect users from phishing attacks on the email level, system
designers and security professionals need to understand how phishers
use persuasion techniques in phishing emails. In this thesis, we present
an analysis of persuasion techniques in phishing emails. Our research
is aimed to understand the characteristics of phishing emails considering
persuasion techniques in the real world analysis which has not been
done yet. 

We have conducted a quantitative analysis on our dataset which consists
of reported phishing emails between August 2013 and December 2013.\foreignlanguage{english}{
The findings are observed from mainly three different viewpoints;
general structural properties, persuasion principles characteristics
and their relationships}. We have found that financial institutions
are the most common target with high number of occurrences in our
dataset. \foreignlanguage{english}{Three important findings of our
research are that (1) authority is the most popular persuasion technique
regardless of the target and the reason used, (2) depending on the
target types and the reason types, the next most popular persuasion
principles are scarcity, consistency, and likeability and (3) scarcity
principle has a high involvement with administrator target type and
account related concerns. }

\vfill{}


\endgroup
